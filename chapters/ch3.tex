% !Mode:: "TeX:UTF-8"

\chapter{表格的绘制方法}
\section{研究生毕业设计论文的绘表规范}

表应有自明性。表格不加左、右边线。表的编排建议采用国际通行的三线表。表内中文书写使用宋体五号字。

每个表格之上均应有表题(由表序和表名组成)。表序一般按章编排,如第~1~章第一个插表的序号为“表~1-1”等。表序与表名之间空两格,
表名使用中文五号字,居中。表名中不允许使用标点符号,表名后不加标点。
表头设计应简单明了,尽量不用斜线。表头中可采用化学,物理量等专业符号。


表中数据应准确无误,书写清楚。数字空缺的格内加横线“-”(占~2~个数字宽度)。表内文字或数字上、下或左、右相同时,
采用通栏处理方式,不允许用“〃”、“同上”之类的写法。

表内文字使用宋体五号字,垂直居中书写,起行空一格、转行顶格、句末不加标点。
如某个表需要转页接排,在随后的各页上应重复表的编号。编号后加“(续表)”,表题可省略。续表应重复表头。
表格绘制完成之后,与正文空一行。

若想获得绘制表格的更多信息,参见网络上的~\href{http://www.tug.org/pracjourn/2007-1/mori/}{Tables in \LaTeXe: Packages and Methods}~文档。

