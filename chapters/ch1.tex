\chapter{内容要求}

\section{题目}
题目应恰当、准确地反映本课题的研究内容。学位论文的中文题目一般不超过25字,不设副标题。
\section{摘要与关键词}
\subsection{摘要}
摘要是论文内容的简要陈述,是一篇具有独立性和完整性的短文。摘要应包括本论文的创造性成果及其理论与实际意义。摘要中不宜使用公式、图表,不标注引用文献编号。避免将摘要写成目录式的内容介绍。
\subsection{关键词}
关键词是供检索用的主题词条,应采用能覆盖论文主要内容的通用技术词条(参照相应的技术术语标准)。关键词一般列3~8个,按词条的外延层次排列(外延大的排在前面)。

\section{论文正文}
论文正文包括引言(或绪论)、论文主体及结论等部分。
\section{引言(或绪论)}
引言(或绪论)一般作为第l章。引言(或绪论)应包括:本研究课题的学术背景及理论与实际意义;国内外文献综述;本研究课题的来源及主要研究内容。
\section{论文主体}
论文主体是学位论文的主要部分,应该结构合理,层次清楚,重点突出,文字简练、通顺。理学、工学的学位论文主体应包括研究内容的总体方案设计及论证、可行性分析、理论分析、实验结果及数据处理分析等。管理学和人文社会学科的论文主体应包括对研究问题的论述及系统分析,比较研究,模型或方案设计,案例论证或实证分析,模型运行的结果分析或建议、改进措施等。\par
工程硕士论文可以是工程设计,也可以是研究论文。学位论文要运用科学的理论方法和技术手段解决工程实际问题,要有新见解。
\section{结论}
学位论文的结论单独作为一章排写,但不加章号。\par
结论是对整个论文主要成果的总结。在结论中应明确指出本研究内容的创造性成果或创新性理论(含新见解、新观点),对其应用前景和社会、经济价值等加以预测和评价,并指出今后进一步在本研究方向进行研究工作的展望与设想。结论内容一般在2000字左右(以汉字计)。
\section{参考文献}
按文中出现的顺序列出直接引用的主要参考文献。博士学位论文参考文献不少于80篇,其中外文文献不少40篇,硕士学位论文参考文献不少于40篇,其中外文文献不少于10篇,专业学位硕士学位论文参考文献不少于30篇,其中外文文献不少于5篇。
\section{攻读学位期间发表的论文}
学位论文后应列出研究生在攻读学位期间发表的(含已录用,并有录用通知书的)与学位论文内容相关的学术论文。
\section{致谢}
对导师和给予指导或协助完成学位论文工作的组织和个人表示感谢。内容应简洁明了、实事求是。对课题给予资助者应予感谢。

