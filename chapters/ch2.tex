% !Mode:: "TeX:UTF-8"

\chapter{数学公式的输入方法}
\section{研究生毕业设计论文的公式规范}

论文中的公式应另起行,原则上应居中书写,与周围文字留有足够的空间区分开。
若公式前有文字(如“解”、“假定”等),文字空两格写,公式仍居中写。公式末不加标点。

公式应标注序号,并将序号置于括号内。 公式序号按章编排,如第~1~章第一个公式序号为“(1-1)”。公式的序号右端对齐。

公式较长时最好在等号“=”处转行,如难实现,则可在~$+$、$-$、$\times$、$\div$~运算符号处转行,转行时运算符号仅书写于转行式前,不重复书写。

文中引用公式时,一般用“见式~(1-1)”或“由公式~(1-1)”。

公式中用斜线表示“除”的关系时应采用括号,以免含糊不清,如~$a/(b\cos x)$。通常“乘”的关系在前,如~$a\cos x/b$而不写成~$(a/b)\cos x$。

不能用文字形式表示等式,如:$\textnormal{刚度}=\frac{{\textnormal{受力}}}{{\textnormal{受力方向的位移}}}$。

对于数学公式的输入方法,网络上有一个比较全面权威的文档\textbf{~\href{http://tug.ctan.org/cgi-bin/ctanPackageInformation.py?id=voss-mathmode}{Math mode}}~请大家事先大概浏览一下。下面将对学位论文中主要用到的数学公式排版形式进行阐述。

\section{生成~\LaTeX~数学公式的两种方法}
对于先前没有接触过~\LaTeX~的人来说,编写~\LaTeX~数学公式是一件很繁琐的事,尤其是对复杂的数学公式来说,更可以说是一件难以完成的任务。
实际上,生成~\LaTeX~数学公式有两种较为简便的方法,一种是基于~MathType~数学公式编辑器的方法,另一种是基于~MATLAB~商业数学软件的方法,
下面将分别对这两种数学公式的生成方法作一下简单介绍。

\subsection{基于~MathType~软件的数学公式生成方法}
MathType~是一款功能强大的数学公式编辑器软件,能够用来在文本环境中插入~Windows OLE~图形格式的复杂数学公式,所以应用比较普遍。但此软件只有~30~天的试用期,之后若再继续使用则需要付费购买才行。网络上有很多破解版的~MathType~软件可供下载免费使用,
笔者推荐下载安装版本号在~6.5~之上的中文破解版。

在安装好~MathType~之后,若在输入窗口中编写数学公式,复制到剪贴板上的仍然是图形格式的对象。
若希望得到可插入到~\LaTeX~编辑器中的文本格式对象,则需要对~MathType~软件做一下简单的设置:在~MathType~最上排的按钮中依次选择“参数选项
$\to$转换”,在弹出的对话窗中选中“转换到其它语言(文字):”,在转换下拉框中选择“Tex~--~--~LaTeX 2.09 and later”,并将对话框最下方的两个复选框全部勾掉,点击确定,这样,再从输入窗口中复制出来的对象就是文本格式的了,就可以直接将其粘贴到~\LaTeX~
编辑器中了。按照这种方法生成的数学公式两端分别有标记\verb|\[|和标记\verb|\]|,在这两个标记之间才是真正的数学公式代码。

若希望从~MathType~输入窗口中复制出来的对象为图形格式,则只需再选中“公示对象(Windows OLE~图形)”即可。

\subsection{基于~MATLAB~软件的数学公式生成方法}

MATLAB~是矩阵实验室(Matrix Laboratory)的简称,是美国~MathWorks~公司出品的商业数学软件。它是当今科研领域最常用的应用软件之一,
具有强大的矩阵计算、符号运算和数据可视化功能,是一种简单易用、可扩展的系统开发环境和平台。

MATLAB~中提供了一个~latex~函数,它可将符号表达式转化为~\LaTeX~数学公式的形式。其语法形式为~latex(s),其中,~s~为符号表达式,
之后再将~latex~函数的运算结果直接粘贴到~\LaTeX~编辑器中。从~\LaTeX~数学公式中可以发现,其中可能包含如下符号组合:

\begin{verbatim*}
\qquad=两个空铅(quad)宽度
\quad=一个空铅宽度
\;=5/18空铅宽度
\:=4/18空铅宽度
\,=3/18空铅宽度
\!=-3/18空铅宽度
\ =一个空格
\end{verbatim*}

所以最好将上述符号组合从数学公式中删除,从而使数学公式显得匀称美观。

对于~Word~等软件的使用者来说,在我们通过~MATLAB~运算得到符号表达式形式的运算结果时,在~Word~中插入运算结果需要借助于~MathType~软件,
通过在~MathType~中输入和~MATLAB~运算结果相对应的数学表达形式,之后再将~MathType~数学表达式转换为图形格式粘贴到~Word~中。实际上,
也可以将~MATLAB~中采用~latex~函数运行的结果直接粘贴到~MathType~中,再继续上述步骤,这样可以大大节省输入公式所需要的时间。
此方法在~MathType~6.5c~上验证通过,若您粘入到~MathType~中的仍然为从~MATLAB~中导入的代码,请您更新~MathType~软件。

\section{数学字体}
在数学模式下,常用的数学字体命令有如下几种:

\begin{verbatim}
\mathnormal或无命令 用数学字体打印文本;
\mathit             用斜体(\itshape)打印文本;
\mathbf             用粗体(\bfseries)打印文本;
\mathrm             用罗马体(\rmfamily)打印文本;
\mathsf             用无衬线字体(\sffamily)打印文本;
\mathtt             用打印机字体(\ttfamily)打印文本;
\mathcal            用书写体打印文本;
\end{verbatim}

在学位论文撰写中,只需要用到上面提到的~\verb|\mathit|、\verb|\mathbf|~和~\verb|\mathrm|~命令。若要得到~Times New Roman~的数学字体,则需要调用~txfonts~宏包(此宏包实际上采用的是~Nimbus Roman No9 L~字体,
它是开源系统中使用的免费字体,其字符字体与~Times New Roman~字体几乎完全相同);若要得到粗体数学字体,则需要调用~bm~宏包。表~\ref{tab:fonts}~中分别列出了得到阿拉伯数字、拉丁字母和希腊字母
各种数学字体的命令。

如果遇到某些符号不知道该采用什么命令能输出它时,则可通过~\href{http://detexify.kirelabs.org/classify.html}{Detexify$^2$~网站}来获取符号命令。若用鼠标左键在此网页的方框区域内画出你所要找的符号形状,则会在网页右方列出和你所画符号形状相近的~5~个符号及其相对应的~\LaTeX~输入命令。若所列出的符号中不包括你所要找的符号,还可通过点击“Select from the complete list!”的链接以得分从低到高的顺序列出所有符号及其相对应的~\LaTeX~输入命令。

